\documentclass[12pt]{article}

\newcommand{\lapl}{\nabla^{2}}
\newcommand{\lr}[1]{\left(#1\right)}
\newcommand{\wh}[1]{\widehat{#1}}
\newcommand{\ua}{\overline{u}}

\usepackage{amsmath, amsfonts}
\usepackage[margin=0.75in]{geometry}

\usepackage[dvipsnames]{xcolor}
\newcommand\myshade{85}
\colorlet{mylinkcolor}{violet}
\colorlet{mycitecolor}{YellowOrange}
\colorlet{myurlcolor}{Aquamarine}
\usepackage{hyperref}
\hypersetup{
    pdfstartview={FitH},    % fits the width of the page to the window
    colorlinks=true,       % false: boxed links; true: colored links
    filecolor=magenta,      % color of file links
    linkcolor=mylinkcolor!\myshade!black,
    citecolor=mycitecolor!\myshade!black,
    urlcolor=myurlcolor!\myshade!black,
}

\title{Thermal Convection Chebyshev Code}
%\author{}

\begin{document}

  \maketitle

  \section{Governing Equations}
    We will solve for the vertical velocity $v\lr{x,y}$, the horizontal velocity $u\lr{x,y} = \ua\lr{y} + u\lr{x,y}$ 
    (decomposed into the horizontal mean $\ua\lr{y}$ and the fluctuations about the mean $u\lr{x,y}$.  We will 
    also ultimately solve for the temperature field $T\lr{x,y}$ which is the sum of the conduction solution and 
    the departure from the conduction solution (see below).  The nondimensional equations are 
    \begin{align}
      \frac{\partial \lapl v}{\partial t} + \partial_{x}\lr{u\lapl v - v\lapl u} = \nu_{0}\lapl\lapl v \\
      \frac{\partial T}{\partial t} + \mathbf{u}\cdot\nabla T = \kappa_{0}\lapl T \\
      \partial_{x}u + \partial_{y}v = 0 \\
      \frac{\partial \ua}{\partial t} + \partial_{y} \overline{uv} = \nu_{0} {\partial_{y}}^{2}  \ua \label{eq:meanflow}.
    \end{align}
%    \begin{align}
%     \frac{\partial \{u}}{\partial t} = \nu_{0} {\partial_{y}}^{2}  \{u} + \partial_{y} \overline{uv}.
%    \end{align}
    with boundary conditions 
    \begin{align}
      &v = \partial_{y} v = 0, \qquad y = \pm 1 \\
      &u\lr{\pm 1} = 0, \quad T\lr{\pm 1} = \mp 1.
    \end{align}
    We will solve for the temperature departure from the conduction state.  In dimensionless units 
    the conduction solution is $T_{c} = -y$.  The departure from the conduction state is 
    $\theta = T + y$.  The equation and boundary conditions governing $\theta$ are 
    \begin{align}
      \frac{\partial \theta}{\partial t} + &\mathbf{u}\cdot\nabla\theta - v = \kappa_{0}\lapl\theta \\
      &\theta\lr{\pm 1} = 0.
    \end{align}
    We shall use $\mathcal{N}$ to denote the nonlinear terms in the temperature equation.

  \section{Spatial Discretization}
  We use a Fourier basis in the $x$-direction and a Chebyshev basis in the $y$-direction.

  \subsection{$x$-Discretization}
    Taking the Fourier transform in the $x$-direction of the temperature equation gives 
    \begin{align}
      \partial_{t}\wh{\theta}_{l} + \wh{\mathcal{N}}_{l} - \wh{v}_{l} = 
        \kappa_{0}\lr{-\lr{\alpha k_{x}}^{2}\wh{\theta}_{l} + \partial_{y}^{2}\wh{\theta}_{l}}
      \label{eq:Tl}
    \end{align}
    where $\alpha k_{x}$ is the wavenumber.

    The continuity equation becomes,
    \begin{align}
      i\lr{\alpha k_{x}}\wh{u}_{l} + \partial_{y}\wh{v} = 0 \label{eq:cont_l}.
    \end{align}

    Note that we do not take the Fourier transform of the mean flow equation because it is
    has no horizontal component.

%Similarly, the mean flow equation becomes,
%\begin{align}
%\partial_{t}\wh{u}_{l} = \nu_{0} {\partial_{y}}^{2} \wh{u}_l - \partial_{y}\wh{{uv}}_l \label{eq:meanflow_l}.
%\end{align}
%
%for $k_x = 0$.

  \subsection{$y$-Discretization}
    We use Chebyshev expansions in the $y$-direction following very closely Wally's 
    Chebyshev approach.  We let 
    \begin{align}
      \partial_{y}^{2}\wh{\theta}_{l} = \sum_{m=0}^{N_{C}-1}{\wh{b}_{ml}T_{m}\lr{y}} = \mathbf{T}_{N}\lr{y}\wh{\mathbf{b}}_{l}.
    \end{align}
    Note that $\mathbf{P}_{T}$ is the projection matrix from physical to Chebyshev space.  
    We have $\mathbf{P}_{T}\mathbf{T}_{N} = \mathbf{I}$.  Following Wally's Chebyshev ``integration'' method 
    (using the hybrid approach) we have 
    \begin{align}
      &\partial_{y}\wh{\theta}_{l} = D\boldsymbol{\Psi}\lr{y}\wh{\mathbf{b}}_{l} \\
      &\wh{\theta}_{l} = \boldsymbol{\Psi}_{N}\lr{y}\wh{\mathbf{b}}_{l}.
    \end{align}
    Note that the matrices $\boldsymbol{\Psi}\lr{y}$ have the boundary conditions 
    built in automatically (check their form!).  These are the Galerkin modes in the 
    code.  

    Similarly, for the velocity fields we have 
    \begin{align}
      \partial_{y}^{4}\wh{v}_{l} &= \mathbf{T}_{N}\lr{y}\wh{\mathbf{a}}_{l} \\
      %\partial_{y}^{2}\wh{u}_{l} &= \mathbf{T}_{N}\lr{y}\wh{\mathbf{d}}_{l} \\
      \partial_{y}^{2}\ua        &= \mathbf{T}_{N}\lr{y}\mathbf{c}.
    \end{align}
    
%    \begin{align}
%     \partial_{y}^{2}\{u} &= \mathbf{T}_{N}\lr{y}\wh{\mathbf{c}}.
%    \end{align}
    
    This gives, 
    \begin{align}
      \partial_{y}^{3}\wh{v}_{l} &= D3\mathbf{V}\lr{y}\wh{\mathbf{a}}_{l} \\
      \partial_{y}^{2}\wh{v}_{l} &= D2\mathbf{V}\lr{y}\wh{\mathbf{a}}_{l} \\
      \partial_{y}\wh{v}_{l} &= D\mathbf{V}\lr{y}\wh{\mathbf{a}}_{l} \\
      \wh{v}_{l} &= \mathbf{V}\lr{y}\wh{\mathbf{a}}_{l}
    \end{align}
    and 
    \begin{align}
      \partial_{y}\ua &= D\mathbf{U}\lr{y}\mathbf{c} \\
                  \ua &= \mathbf{U}\lr{y}\mathbf{c}.
    \end{align}
%    and 
%    \begin{align}
%      &\partial_{y}\wh{u}_{l} = D\mathbf{U}\lr{y}\wh{\mathbf{d}}_{l} \\
%      &\wh{u}_{l} = \mathbf{U}\lr{y}\wh{\mathbf{d}}_{l}.
%    \end{align}
    where, once again, the Galerkin modes $\mathbf{V}$ and $\mathbf{U}$ 
    have the appropriate boundary conditions built in.

%    Also,
%    
%    \begin{align}
%    &\partial_{y}\{u} = D\mathbf{U}\lr{y}\wh{\mathbf{c}} \\
%    &\{u} = \mathbf{U}\lr{y}\wh{\mathbf{c}}.
%    \end{align}

    Using the Chebyshev expansions in~\eqref{eq:Tl} gives 
    \begin{align}
      \boldsymbol{\Psi}\lr{y}\partial_{t}\wh{\mathbf{b}}_{l} + \wh{\mathcal{N}}_{l} - 
      \mathbf{V}\lr{y}\wh{\mathbf{a}}_{l} = 
      \kappa_{0}\lr{-\lr{\alpha k_{x}}^{2}\boldsymbol{\Psi}\lr{y}\wh{\mathbf{b}}_{l} + 
                    \mathbf{T}\lr{y}\wh{\mathbf{b}}_{l}}.
    \end{align}
    Finally, we project into Chebyshev space, 
    \begin{align}
      \mathbf{P}_{T}\boldsymbol{\Psi}\lr{y}\partial_{t}\wh{\mathbf{b}}_{l} + \mathbf{P}_{T}\wh{\mathcal{N}}_{l} - 
      \mathbf{P}_{T}\mathbf{V}\lr{y}\wh{\mathbf{a}}_{l} = 
      \kappa_{0}\lr{-\lr{\alpha k_{x}}^{2}\mathbf{P}_{T}\boldsymbol{\Psi}\lr{y} + \mathbf{I}}\wh{\mathbf{b}}_{l}.
    \end{align}

    Note that we will only use the continuity equation to solve for the horizontal velocity 
    perturabation.  Further note that the horizontal velocity perturbation only appears in 
    the nonlinear terms in each equation.  The nonlinear terms are computed in physical space 
    (since we're using a pseudospectral method) and so we never actually need the horizontal 
    velocity perturbation in Chebyshev space.  With this in mind, we proceed to compute the 
    relevant derivatives of the horizontal velocity perturbation that will be used in the 
    nonlinear terms in each of the main evolution equations (temperature, vertical velocity, 
    and mean flow).  We have, 
    \begin{align}
      \partial_{x}u + \partial_{y}v = 0.
    \end{align}
    Taking the Fourier transform gives 
    \begin{align}
      \wh{u}_{l}\lr{y} = \frac{i}{\alpha k_{x}} \partial_{y}\wh{v}_{l}\lr{y}.
    \end{align}
    Expanding the $y$-derivative of $v$ in Chebyshev modes leads to,
    \begin{align}
      \wh{u}_{l}\lr{y} = \frac{i}{\alpha k_{x}} D\mathbf{V}\lr{y}\wh{\mathbf{a}}_{l}.
      \label{eq:uly}
    \end{align}
    We will also need the first derivative in $y$ of the horizontal velocity perturbation.  
    We begin by taking the $y$-derivative of the continuity equation, 
    \begin{align}
      \partial_{x}\partial_{y}u + \partial_{y}^{2}v = 0.
    \end{align}
    After taking the Fourier transform we get, 
    \begin{align}
      \partial_{y}\wh{u}_{l} = \frac{i}{\alpha k_{x}}\partial_{y}^{2}\wh{v}_{l}.
    \end{align}
    Expanding in Chebyshev modes results in 
    \begin{align}
      \partial_{y}\wh{u}_{l}\lr{y} = \frac{i}{\alpha k_{x}}D2\mathbf{V}\lr{y}\wh{\mathbf{a}}_{l}.
      \label{eq:dyuly}
    \end{align}

%    Using the Chebyshev discretization in the continuity equation~\eqref{eq:cont_l} gives 
%    \begin{align}
%      i\lr{\alpha k_{x}}\mathbf{U}\wh{\mathbf{d}}_{l} + D\mathbf{V}\wh{\mathbf{a}}_{l} = 0.
%    \end{align}
%    Projecting into Chebyshev space results in 
%    \begin{align}
%      i\lr{\alpha k_{x}}\mathbf{P}_{T}\mathbf{U}\wh{\mathbf{d}}_{l} + \mathbf{P}_{T}D\mathbf{V}\wh{\mathbf{a}}_{l} = 0.
%    \end{align}
%    When $k_{x} \neq 0 $ we can solve for the horizontal velocity modes $\displaystyle \wh{\mathbf{d}}_{l}$. 
%    When $k_{x} = 0$ we need to write a separate equation for the mean flow.

    The mean flow equation~\eqref{eq:meanflow} using the Chebyshev expansions becomes,
    \begin{align}
      \mathbf{U}\lr{y}\partial_{t}\mathbf{c} + \mathcal{N}_{u} = \nu_{0}\mathbf{T}_N\lr{y}\mathbf{c}
    \end{align} 
    where  $\mathcal{N}_{u} = \partial_{y}\overline{uv}$.
    
    After projection into Chebyshev space, we have,
    \begin{align}
      \mathbf{P}_{T}\mathbf{U}\lr{y}\partial_{t}\mathbf{c} + \mathbf{P}_{T}\mathcal{N}_{u} = 
      \nu_{0}\mathbf{I}\mathbf{c} .
    \end{align}

    Our notation regarding the nonlinear term is a bit strange.  However, we shall discuss
    how this term is computed which will shed some light on our notation of choice.  First, 
    we rewrite the nonlinear term as 
    \begin{align}
      \partial_{y}\overline{uv} = \overline{u\partial_{y}v} + \overline{v\partial_{y}u}.
    \end{align}
    Note that in the code each term is in Fourier space.  Here are some steps for computing the 
    nonlinear term:
    \begin{enumerate}
      \item The first step is to inverse transform from wavenumber space to $x$-space.  
      \item Then, using~\eqref{eq:uly} and~\eqref{eq:dyuly} we can bring each term into physical space.  
      \item Next compute the average in the $x$-direction.  
      \item At this point, we have computed the nonlinear term in physical space.  The last step is to 
            bring it back into Chebyshev space.  Note that it is not necessary to do a Fourier transform 
            because we have averaged the horizontal direction out.
    \end{enumerate}

    %\begin{align}
    %\boldsymbol{\mathbf{U}}\lr{y}\partial_{t}\wh{\mathbf{c}} + \wh{\mathcal{N}}_u = 
    %\mathbf{T}_N\lr{y}\wh{\mathbf{c}} .
    %\end{align} 
    %
    %where  $\wh{\mathcal{N}}_u =   \partial_{y} \overline{uv}$.  This term is evaluated as follows,
    %\begin{align}
    %\overline{uv} =  \mathbf{T}_N\lr{y}\wh{\mathbf{n}} \\
    %\partial_{y} \overline{uv} = \mathbf{T}_N\lr{y}\wh{\mathbf{m}}.
    %\end{align}
    %where $\wh{\mathbf{m}}$ is computed using 'C2Cder' subroutine (Cheb to Cheb differentiation) from Wally's notes.
    %
    %After projection into Chebyshev space, we have,
    %\begin{align}
    %\mathbf{P}_{T}\boldsymbol{\mathbf{U}}\lr{y}\partial_{t}\wh{\mathbf{c}} + \mathbf{I}\wh{m} = 
    %\mathbf{I}\wh{\mathbf{c}} .
    %\end{align}

  \section{Temporal Discretization}

  \section{Verification and Validation (V\&V)}
  In this section we summarize some of the tests that we performed to verify our code.  We 
  primarily focus on code verification at this point rather than validation.  At some point, 
  we would need to do some validation testing as well.

  \subsection{Mean-Flow Implementation}
    We ran a simulation with $\ua\lr{y,t=0} = 0$ and turned off the nonlinear term.  The 
    test here is to ensure that no mean flow develops.  If a mean flow does develop then 
    we have inadvertently introduced a source into the mean flow equation.  When running 
    this test, the mean flow remained zero for all time.

    The next set of tests involved running a simulation with $\ua\lr{y,t=0} = \ua_{0}\lr{y}$ 
    and no nonlinear term.  In this situation, we have an analytical solution and we can 
    compare our analytical solution to the numerical solution.  We also do a convergence 
    study.  First we did a mesh refinement by factors of $2$ in the $y$-direction and 
    then we did a mesh refinement by factors of $2$ in the $x$-direction.  In each case 
    we plotted the $L_{2}$ error as a function of mesh parameter to make sure that we 
    achieved the correct convergence rate.

\end{document}
