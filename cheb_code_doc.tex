\documentclass[12pt]{article}

\newcommand{\lapl}{\nabla^{2}}
\newcommand{\lr}[1]{\left(#1\right)}
\newcommand{\wh}[1]{\widehat{#1}}

\usepackage{amsmath, amsfonts}
\usepackage[margin=0.75in]{geometry}

\usepackage[dvipsnames]{xcolor}
\newcommand\myshade{85}
\colorlet{mylinkcolor}{violet}
\colorlet{mycitecolor}{YellowOrange}
\colorlet{myurlcolor}{Aquamarine}
\usepackage{hyperref}
\hypersetup{
    pdfstartview={FitH},    % fits the width of the page to the window
    colorlinks=true,       % false: boxed links; true: colored links
    filecolor=magenta,      % color of file links
    linkcolor=mylinkcolor!\myshade!black,
    citecolor=mycitecolor!\myshade!black,
    urlcolor=myurlcolor!\myshade!black,
}

\title{Thermal Convection Chebyshev Code}
%\author{}

\begin{document}

  \maketitle

  \section{Governing Equations}
    The nondimensional equations are 
    \begin{align}
      \frac{\partial \lapl v}{\partial t} + \partial_{x}\lr{u\lapl v - v\lapl u} = \nu_{0}\lapl\lapl v \\
      \frac{\partial T}{\partial t} + \mathbf{u}\cdot\nabla T = \kappa_{0}\lapl T \\
      \partial_{x}u + \partial_{y}v = 0.
    \end{align}
    with boundary conditions 
    \begin{align}
      &v = \partial_{y} v = 0, \qquad y = \pm 1 \\
      &u\lr{\pm 1} = 0, \quad T\lr{\pm 1} = \mp 1.
    \end{align}
    We will solve for the temperature departure from the conduction state.  In dimensionless units 
    the conduction solution is $T_{c} = -y$.  The departure from the conduction state is 
    $\theta = T + y$.  The equation and boundary conditions governing $\theta$ are 
    \begin{align}
      \frac{\partial \theta}{\partial t} + &\mathbf{u}\cdot\nabla\theta - v = \kappa_{0}\lapl\theta \\
      &\theta\lr{\pm 1} = 0.
    \end{align}
    We shall use $\mathcal{N}$ to denote the nonlinear terms in the temperature equation.

  \section{Spatial Discretization}
  We use a Fourier basis in the $x$-direction and a Chebyshev basis in the $y$-direction.

  \subsection{$x$-Discretization}
    Taking the Fourier transform in the $x$-direction of the temperature equation gives 
    \begin{align}
      \partial_{t}\wh{\theta}_{l} + \wh{\mathcal{N}}_{l} - \wh{v}_{l} = 
        \kappa_{0}\lr{-\lr{\alpha k_{x}}^{2}\wh{\theta}_{l} + \partial_{y}^{2}\wh{\theta}_{l}}
      \label{eq:Tl}
    \end{align}
    where $\alpha k_{x}$ is the wavenumber.

    The continuity equation becomes,
    \begin{align}
      i\lr{\alpha k_{x}}\wh{u}_{l} + \partial_{y}\wh{v} = 0 \label{eq:cont_l}.
    \end{align}

  \subsection{$y$-Discretization}
    We use Chebyshev expansions in the $y$-direction following very closely Wally's 
    Chebyshev approach.  We let 
    \begin{align}
      \partial_{y}^{2}\wh{\theta}_{l} = \sum_{m=0}^{N_{C}-1}{\wh{b}_{ml}T_{m}\lr{y}} = \mathbf{T}_{N}\lr{y}\wh{\mathbf{b}}_{l}.
    \end{align}
    Note that $\mathbf{P}_{T}$ is the projection matrix from physical to Chebyshev space.  
    We have $\mathbf{P}_{T}\mathbf{T}_{N} = \mathbf{I}$.  Following Wally's Chebyshev ``integration'' method 
    (using the hybrid approach) we have 
    \begin{align}
      &\partial_{y}\wh{\theta}_{l} = D\boldsymbol{\Psi}\lr{y}\wh{\mathbf{b}}_{l} \\
      &\wh{\theta}_{l} = \boldsymbol{\Psi}_{N}\lr{y}\wh{\mathbf{b}}_{l}.
    \end{align}
    Note that the matrices $\boldsymbol{\Psi}\lr{y}$ have the boundary conditions 
    built in automatically (check their form!).  These are the Galerkin modes in the 
    code.  

    Similarly, for the velocity fields we have 
    \begin{align}
      \partial_{y}^{4}\wh{v}_{l} &= \mathbf{T}_{N}\lr{y}\wh{\mathbf{a}}_{l} \\
      \partial_{y}^{2}\wh{v}_{l} &= \mathbf{T}_{N}\lr{y}\wh{\mathbf{d}}_{l}.
    \end{align}
    This gives, 
    \begin{align}
      \partial_{y}^{3}\wh{v}_{l} &= D3\mathbf{V}\lr{y}\wh{\mathbf{a}}_{l} \\
      \partial_{y}^{2}\wh{v}_{l} &= D2\mathbf{V}\lr{y}\wh{\mathbf{a}}_{l} \\
      \partial_{y}\wh{v}_{l} &= D\mathbf{V}\lr{y}\wh{\mathbf{a}}_{l} \\
      \wh{v}_{l} &= \mathbf{V}\lr{y}\wh{\mathbf{a}}_{l}
    \end{align}
    and 
    \begin{align}
      &\partial_{y}\wh{u}_{l} = D\mathbf{U}\lr{y}\wh{\mathbf{d}}_{l} \\
      &\wh{u}_{l} = \mathbf{U}\lr{y}\wh{\mathbf{d}}_{l}.
    \end{align}
    where, once again, the Galerkin modes $\mathbf{V}$ and $\mathbf{U}$ 
    have the appropriate boundary conditions built in.

    Using the Chebyshev expansions in~\eqref{eq:Tl} gives 
    \begin{align}
      \boldsymbol{\Psi}\lr{y}\partial_{t}\wh{\mathbf{b}}_{l} + \wh{\mathcal{N}}_{l} - 
      \mathbf{V}\lr{y}\wh{\mathbf{a}}_{l} = 
      \kappa_{0}\lr{-\lr{\alpha k_{x}}^{2}\boldsymbol{\Psi}\lr{y}\wh{\mathbf{b}}_{l} + 
                    \mathbf{T}\lr{y}\wh{\mathbf{b}}_{l}}.
    \end{align}
    Finally, we project into Chebyshev space, 
    \begin{align}
      \mathbf{P}_{T}\boldsymbol{\Psi}\lr{y}\partial_{t}\wh{\mathbf{b}}_{l} + \mathbf{P}_{T}\wh{\mathcal{N}}_{l} - 
      \mathbf{P}_{T}\mathbf{V}\lr{y}\wh{\mathbf{a}}_{l} = 
      \kappa_{0}\lr{-\lr{\alpha k_{x}}^{2}\mathbf{P}_{T}\boldsymbol{\Psi}\lr{y} + \mathbf{I}}\wh{\mathbf{b}}_{l}.
    \end{align}

    Using the Chebyshev discretization in the continuity equation~\eqref{eq:cont_l} gives 
    \begin{align}
      i\lr{\alpha k_{x}}\mathbf{U}\wh{\mathbf{d}}_{l} + D\mathbf{V}\wh{\mathbf{a}}_{l} = 0.
    \end{align}
    Projecting into Chebyshev space results in 
    \begin{align}
      i\lr{\alpha k_{x}}\mathbf{P}_{T}\mathbf{U}\wh{\mathbf{d}}_{l} + \mathbf{P}_{T}D\mathbf{V}\wh{\mathbf{a}}_{l} = 0.
    \end{align}
    When $k_{x} \neq 0 $ we can solve for the horizontal velocity modes $\displaystyle \wh{\mathbf{d}}_{l}$. 
    When $k_{x} = 0$ we need to write a separate equation for the mean flow.

  \section{Temporal Discretization}

\end{document}
